\documentclass[final]{beamer}

\usepackage{times}
\usepackage{amsmath,amsthm, amssymb}
\usepackage{wrapfig}
\usepackage{tcolorbox}
\usepackage[orientation=landscape,size=a0,scale=1.5]{beamerposter}
\usepackage{tikz}
\usetikzlibrary{shadows.blur} %!!!
\usetikzlibrary{shadows}
\usetheme{gemini}
\usepackage{xcolor}
\usepackage{relsize}


\graphicspath{{./Images/}}

\definecolor{customcolor}{HTML}{D0E0E3}
\definecolor{titlebgcolor}{HTML}{2F565D}%{FE9090}%{E06666}
\definecolor{myblockcolor}{HTML}{5093A2}
\definecolor{blockcolor}{HTML}{FFE6CF}%{FFDADA}%{F9CB9C}

\usecolortheme{whale}
\setbeamercolor{background canvas}{bg=customcolor}
%\setbeamercolor{titlelike}{parent=structure,bg=titlebgcolor}

%\setbeamercolor{titlelike}{parent=structure,bg=cyan}

% If you have N columns, choose \sepwidth and \colwidth such that
% (N+1)*\sepwidth + N*\colwidth = \paperwidth
\newlength{\sepwidth}
\newlength{\colwidth}
\setlength{\sepwidth}{0.025\paperwidth}
\setlength{\colwidth}{0.3\paperwidth}

\newcommand{\separatorcolumn}{\begin{column}{\sepwidth}\end{column}}
\newcommand{\placetextbox}[4]{% \placetextbox{<offset top>}{<offset left/right>}{<align>}{<stuff>}
  \setbox0=\hbox{#4}% Put <stuff> in a box
  \AddToShipoutPictureFG*{% Add <stuff> to current page foreground
    \if#3r
    \put(\LenToUnit{\paperwidth-#1},\LenToUnit{\paperheight-#2}){\vtop{{\null}\makebox[0pt][r]{\begin{tabular}{r}#4\end{tabular}}}}%
    \else
    \put(\LenToUnit{#1},\LenToUnit{\paperheight-#2}){\vtop{{\null}\makebox[0pt][l]{\begin{tabular}{l}#4\end{tabular}}}}%
    \fi
  }%
}%

\newtcolorbox{myblock}[2][]{
  title=\centering\Large\bfseries #2,
  colback=blockcolor,
  colbacktitle=myblockcolor,
  coltitle=white,
  fonttitle=\bfseries,
  colframe=myblockcolor,
  boxrule=0pt,
  toprule=2pt,
  toptitle=5mm,
  bottomtitle=5mm,
  titlerule=0mm,
  arc=5mm,
  outer arc=5mm,
  leftrule=0pt,
  rightrule=0pt,
  bottomrule=0pt,
  boxsep=0pt,
  left=10mm,
  right=10mm,
  top=10mm,
  bottom=10mm,
  shadow={80mm}{-80mm}{90mm}{black!90!white,blur},
  drop shadow=black!90!white,
  enhanced,
}

%   Logos ======================================================
\logoleft{\includegraphics[height=7.5cm]{DUlogo.png}}
% \vspace{8cm}
\logoright{\includegraphics[width=20cm]{MISCADA.jpg}}

%   Body =========================================================
\begin{document}

\begin{frame}{}
\begin{tikzpicture}[remember picture,overlay]
\node [below = 2cm] at (current page.north) [anchor=north,inner sep=0pt] 
  {
    \begin{minipage}{\textwidth}
      \centering
      \begin{tikzpicture}
        \node[rounded corners=10pt, drop shadow, fill=titlebgcolor, text width=0.6\textwidth, minimum height=10cm, align=center, text=white]
        (titlebox) { 
          \bfseries\Huge De Bono's Thinking Hats \\[1cm] 
          \normalsize - An Innovation Method - \\
          \normalfont H. Fullwood, S. Hasan \\ 
           
        };
      \end{tikzpicture}
    \end{minipage}
  };
\end{tikzpicture}

\begin{columns}[t]
\separatorcolumn
%   Column 1 ======================================================
\begin{column}{\colwidth}
    \vspace{-2cm}
    \begin{myblock}{Overview}
    De Bono’s Thinking hats, also known as the Six Thinking Hats, are a way for companies to incorporate different considerations into their thinking process. It is a simple and effective thinking process that can boost people's moral, productivity and focus towards the given task.
    \end{myblock}

    \begin{myblock}{The White Hat} 
    \begin{figure}
        \includegraphics[height=0.4\textwidth]{Images/white_tr.png}
    \end{figure}
    \textbf{Facts, Objective, Information gathering or Data}
    \\ What information is available?
    What are the facts we have and what do we lack?
    What does it tell us?
    How are we going to get the information?
    \\ $\looparrowright$ For example, how information can help to tackle a particular issue.

    \end{myblock}

    \begin{myblock}{The Red Hat}
    \begin{figure}
        \includegraphics[height=0.4\textwidth]{Images/red_tr.png}
    \end{figure}
    \textbf{Emotion, Feelings, Hunches, Intuition}

    \\ How do you feel about the situation?
    % 'I just don't think that idea will work.'

    \\ $\looparrowright$ For example considerations about: environment, ethics, working environment, effects on current world issues.
    \end{myblock}
  
\end{column}
%   Column 2 ======================================================
\separatorcolumn
\begin{column}{\colwidth}
    \vspace{-2cm}
    \begin{myblock}{The Yellow Hat} 
    \begin{figure}
        \includegraphics[height=0.4\textwidth]{Images/yellow_tr.png}
    \end{figure}
    \textbf{Benefits, Positiveness, Feasibility}
    \\ What are the advantages of applying the solution?
    Why do you think it is workable?
    
    \\ $\looparrowright$ For example, consideration about the benefits of a new idea or a decision and how feasible this would be.

    \end{myblock}

    \begin{myblock}{The Green Hat} 
    \begin{figure}
        \includegraphics[height=0.4\textwidth]{Images/green_tr.png}
    \end{figure}
    \textbf{Creativity, Opportunity, Alternatives}

    \\ What could be another solution?
    What other ideas do you have?
    % What about approaching the issue from the opposite viewpoint?

    \\ $\looparrowright$ For example, potential new partnerships with different companies, targeting different markets or better solutions to old problems.
    \end{myblock}

    \begin{myblock}{The Forgotten Hat} 
    There is actually a seventh proposed hat known as the grey hat, it is meant to represent wisdom and some innovative schemes use this as a seventh hat - despite it not being widely accepted.
    \end{myblock}
    
\end{column}
%   Column 3 ======================================================
\separatorcolumn
\begin{column}{\colwidth}
    \vspace{-2cm}

    \begin{myblock}{The Black Hat} 
    \begin{figure}
        \includegraphics[height=0.4\textwidth]{Images/black_tr.png}
    \end{figure}
    \textbf{Risks, Negativity, Impact, Caution}
    \\ What are the risks?
    What are the worst-case scenarios?
    Why could the solution not work?
    \\ $\looparrowright$ For example, identify any weak points in an idea or particular decision, and work out how to avoid them.
    \end{myblock}

    
    \begin{myblock}{The Blue Hat} 
    \begin{figure}
        \includegraphics[height=0.42\textwidth]{Images/tr2.png}
    \end{figure}
    \textbf{Manages the realisations of all of the other hats.}

    Have you considered all aspects equally?
    How would you summarize all of the information from the other hats?

    \\ $\looparrowright$ For example, balancing the risk vs rewards of new ideas whilst maintaining morality and profits.
    \end{myblock}

    \begin{myblock}{Conclusion} 
    The accumulation of these different strains of thinking can be incredibly beneficial to a company’s innovation. Benefits include: maximising productivity, making clear and objective decisions and also allows the consideration of the problem as a whole.
    \end{myblock}
    
\end{column}

\separatorcolumn
\end{columns}
\end{frame}
\end{document}
%    \vfill
%    \begin{block}{\large Fontsizes}
%      \centering
%      {\tiny tiny}\par
%      {\scriptsize scriptsize}\par
%      {\footnotesize footnotesize}\par
%      {\normalsize normalsize}\par
%      ...
%    \end{block}
%    
%    \vfill
%    \begin{columns}[t]
%      \begin{column}{.30\linewidth}
%        \begin{block}{Introduction}
%          \begin{itemize}
%%          \item some items
%          \item some items
%%          ...
 %         \end{itemize}
 %       \end{block}
 %     \end{column}
 %     \begin{column}{.48\linewidth}
 %       \begin{block}{Introduction}
 %         \begin{itemize}
 %         \item some items and $\alpha=\gamma, \sum_{i}$
 %         ...
 %         \end{itemize}
 %         $$\alpha=\gamma, \sum_{i}$$
 %       \end{block}
  %      ...

%      \end{column}
%    \end{columns}
%  \end{frame}
